\documentclass[a4paper, 11pt]{article}
\usepackage[utf8x]{inputenc} 
\usepackage[T1]{fontenc}
\usepackage{lmodern}
\usepackage{graphicx}
\usepackage[french]{babel}
\usepackage[round]{natbib}
\usepackage{tikz}
\usetikzlibrary{positioning,arrows,3d,calc,shapes,fit,shapes.geometric,automata,arrows.meta}
\usepackage{minted}

\newtheorem{exemple}{Exemple}

\title{Spécification et implémentation de comportement d'agent}

\date{\today}
\author{Quentin Baert, Maxime Morge\\
Univ. Lille, CNRS, Centrale Lille\\
UMR 9189 - CRIStAL\\
F-59000 Lille, France\\
Email: \{Quentin.Baert,Maxime.Morge\}@univ-lille.fr
}

\begin{document}
\maketitle

L'approche centrée individu nécessite la mise en œuvre d'agents
autonomes. Inspiré du modèle d'acteurs de \cite{hewitt77aij}, nous
considérons qu'un agent~:
\begin{enumerate}
\item possède une adresse unique pour communiquer avec ses pairs par
  envoi de messages~;
\item peut créer d'autres agents ;
\item est muni d'un état mental ;
\item réagit aux messages reçus selon un comportement.
\end{enumerate}

\textbf{Message.} Un agent possède une file d'attente qui contient
l'ensemble des messages qu'il a reçus. Nous considérons ici une
version simplifiée du modèle de transmission asynchrone de messages
pour la programmation concurrente~\citep{clinger81foundations}. En
d'autres termes, nous suppose que~:
\begin{enumerate}
\item le délai de transmission des messages est arbitraire
mais non négligeable ;
\item l'ordre d'émission/réception des messages est identique par pair\\
  émetteur-récepteur ;
\item la distribution des messages est garantie.
\end{enumerate}
Contrairement au modèle
de~\cite{clinger81foundations}, nous faisons l'hypothèse que les
messages sont livrés exactement un fois.

\textbf{Création d'agents.} Un agent peut être composé de plusieurs
agents composants avec lesquels il communique. Une telle architecture
composite d'agent permet d'isoler les différentes activités d'un agent
pour ainsi faciliter sa conception, l'intelligibilité de son
comportement et sa mise en œuvre.

\textbf{État mental.} Un agent est muni d'une base de connaissances
et/ou de croyances. Cette mémoire constitue une représentation
explicite, locale et subjective de la perception par l'agent du
système multi-agents mais également de son environnement physique. En
particulier, la liste de ses accointances est la liste des pairs avec
lesquels l'agent est capable de communiquer directement par envoi de
message.

\textbf{Comportement.} Pour déterminer sa prochaine action, un agent
retire le premier message de sa file d'attente et réagit en fonction
de son état courant.

Même si tous les agents utilisent le même comportement, chaque agent
est indépendant. D'une part, chacun possède sa propre file de
messages, son propre état courant et son propre état mental. Ces
éléments sont inaccessibles directement par les autres agents : c'est
le principe d'encapsulation. D'autre part, la mise en œuvre
indépendante, concurrente et asynchrone de ces comportements permet au
système d'exécuter l'algorithme multi-agents de manière décentralisée.

\subsection*{Spécification d'un comportement d'agent}

Le comportement d'un agent
peut être décrit par un automate fini déterministe (en anglais,
\textit{finite state machine}) qui spécifie comment l'agent doit
réagir au premier message de sa file en fonction de son état courant
et de son état mental (cf. figure~\ref{fig:comportement}).

Les actions possibles d'un agent sont :
\begin{itemize}
\item l'exécution d'actions sur l'environnement ;
\item la création d'un autre agent ;
\item la mise à jour de son état mental ;
\item l'envoi de messages (dénoté par l'opérateur $\texttt{!}$).
\end{itemize}
Un agent peut envoyer un message à l'un de ses pairs ou à lui-même
(dénoté par la suite \texttt{self}). Dans ce dernier cas, un agent est
en mesure de déclencher son propre comportement.

\begin{figure}[h!]
  \centering
    \centering
  \begin{tikzpicture}[->,>=stealth',shorten >=1pt,auto,
    node distance=2.5cm,scale=.8,transform shape,
    align=center,state/.style={circle,draw,minimum size=2cm,thick,fill=gray!10}]

    \node[state,initial] (s1) {État 1};
    \node[state,right of=s1,xshift=8cm] (s2) {État 2};

    \draw (s1) edge[above] node{$\texttt{émetteur:message if condition} \Rightarrow \texttt{actions}$} (s2);
  \end{tikzpicture}
  \caption{Automate fini déterministe spécifiant le comportement d'un
    agent.  Quand il reçoit un $\texttt{message}$ de la part d'un
    $\texttt{émetteur}$ dans l'$\texttt{état 1}$ et que la
    $\texttt{condition}$ est remplie, l'agent effectuent les
    $\texttt{actions}$ et passe dans l'$\texttt{état 2}$.}
  \label{fig:comportement}
\end{figure}


\begin{exemple}[Spécification d'un comportement d'agent]
  \label{ex:hulk}
  L'automate\\ de la figure~\ref{fig:hulk} spécifie le comportement de
  l'agent Bruce lors de ses interactions avec un agent
  $\texttt{reckless}$ (téméraire). L'agent Bruce possède un état
  mental constité d'une unique variable $\texttt{patience}$ qui
  représente son niveau de patience. L'état initial de l'agent Bruce
  est l'état \emph{Banner}. Tant qu'il lui reste de la
  $\texttt{patience}$, l'agent Bruce accepte de recevoir le message
  $\texttt{Flick}$ (pichenette) de la part de l'agent
  $\texttt{reckless}$ mais perd progressivement
  $\texttt{patience}$. En revanche, lorsque sa $\texttt{patience}$ est
  nulle et qu'il reçoit une $\texttt{Flick}$, l'agent Bruce passe dans
  l'état \emph{Hulk} et envoie un message $\texttt{Warning}$ (avertissement) à
  l'agent $\texttt{reckless}$. Dans l'état \emph{Hulk}, l'agent Bruce
  répond systématiquement à une $\texttt{Flick}$ par un
  $\texttt{HulkSmash}$. Après avoir délivré un $\texttt{HulkSmash}$,
  l'agent Bruce s'envoie un message $\texttt{Calm}$ qui, lorsqu'il
  le traite, déclenche la réinitialisation de sa variable
  $\texttt{patience}$ et l'agent repasse dans l'état \emph{Banner}.
\end{exemple}

\begin{figure}[h!]
  \centering
  % Copyright (C) Quentin BAERT and Maxime MORGE 2019
\documentclass[crop,tikz,
convert={outext=.svg,command=\unexpanded{pdf2svg \infile\space\outfile}},
multi=false]{standalone}
\usetikzlibrary{positioning,arrows,3d,calc,shapes,fit,shapes.geometric,automata}
\begin{document}
% Copyright (C) Quentin BAERT and Maxime MORGE 2019
\documentclass[crop,tikz,
convert={outext=.svg,command=\unexpanded{pdf2svg \infile\space\outfile}},
multi=false]{standalone}
\usetikzlibrary{positioning,arrows,3d,calc,shapes,fit,shapes.geometric,automata}
\begin{document}
% Copyright (C) Quentin BAERT and Maxime MORGE 2019
\documentclass[crop,tikz,
convert={outext=.svg,command=\unexpanded{pdf2svg \infile\space\outfile}},
multi=false]{standalone}
\usetikzlibrary{positioning,arrows,3d,calc,shapes,fit,shapes.geometric,automata}
\begin{document}
\input{bruce.tkiz}
\end{document}
\end{document}
\end{document}
  \caption{Spécification du comportement de l'agent Bruce issu de
    l'exemple~\ref{ex:hulk}.}
  \label{fig:hulk}
\end{figure}

Les automates permettent une spécification complète et formelle du
comportement des agents.

\subsection*{Implémentation d'un comportement d'agent}

Les comportements d'agents peuvent être implémentés grâce à la boîte à
outils Akka~\citep{akka} qui permet la construction d'applications
réactives, concurrentes et distribuées.

Pour illustrer la relation directe entre la spécification d'un
comportement par un automate fini détermiste et son implémentation en
grâce à la boîte à outils Akka, la figure~\ref{fig:hulkcode} présente
l'implémentation du comportement de l'agent Bruce de
l'exemple~\ref{ex:hulk}. Selon l'API Akka, La classe
\texttt{ActorSystem} permet de déployer un système multi-agents. Comme
évoquée à la ligne 26, la classe \texttt{Actor} définit les
méthodes/propriétés d'un agent et la classe \texttt{FSM} permet
décrire son comportement grâce à un automate fini déterministe (en
anglais, \textit{finite state machine}). D'abord est énumérée la liste
exhaustive des types de messages échangés (lignes 8 à 12) puis celle
des états courants possibles pour l'agent (lignes 17 à 19). L'état
mental de l'agent contient l'unique variable \texttt{patience}
initialisée à $3$ (ligne 25). Un agent \texttt{Bruce} est associé à un
état courant et à un état mental (ligne 30). L'agent est initialisé
(ligne 52) dans l'état courant \texttt{Banner} avec un nouvel état
mental \texttt{Mind} (ligne 3).  Dans cet état courant (ligne 35), il
peut recevoir un message \texttt{Flick} : soit sa patience est
positive et il reste dans le même état en décrémentant sa patience
(ligne 36-38) ; soit sa patience est nulle et il passe dans l'état
\texttt{Hulk} en émettant le message \texttt{Warning} (ligne
39-41). Dans l'état courant \texttt{Hulk}, l'agent peut recevoir un
message \texttt{Flick} ou \texttt{Calm}. Dans le premier cas (lignes
46-50), il reste dans le même état courant en répondant par le message
\texttt{HulkSmash} et en s'envoyant à lui-même le message
\texttt{Calm}. Dans le second cas (ligne 51-452), l'agent retourne dans
l'état courant \texttt{Banner} avec un état mental réinitialisé.

\begin{figure}[htbp]
  \inputminted[breaklines,linenos,fontsize=\scriptsize]{scala}{../src/main/scala/cristalsmac/mas4data/sample/Hulk.scala}
  \caption{Code qui implément le comportement illustré par la
    figure~\ref{fig:hulk}.}
  \label{fig:hulkcode}
\end{figure}


En résumé l'automate qui spécifie le comportement d'un agent
s'implémente naturellement et directement grâce à Akka.

\bibliographystyle{plainnat}
\bibliography{hulkDocumentation}

\end{document}